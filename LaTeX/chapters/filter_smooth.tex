\section{Filtering and Marginal Likelihood Computation}

Assume that we are interested in the sequential approximation of the joint posterior distributions
\[
	\{p(x_{1:n} \mid y_{1:n})\}_{n \ge 1},
\]
as well as the corresponding marginal likelihoods
\[
	\{p(y_{1:n})\}_{n \ge 1}.
\]
That is, at the first time instance we wish to approximate \(p(x_1 \mid y_1)\) and \(p(y_1)\); at the second time instance, \(p(x_{1:2} \mid y_{1:2})\) and \(p(y_{1:2})\); and so on. We refer to this problem as the \emph{optimal filtering problem}. Note that this definition is slightly different from the usage in much of the literature, where the term \emph{filtering} is often reserved for estimating the marginal distributions
\[
	\{p(x_n \mid y_{1:n})\}_{n \ge 1},
\]
rather than the joint distributions
\[
	\{p(x_{1:n} \mid y_{1:n})\}_{n \ge 1}.
\]

\section{Smoothing}

Consider now the problem of sampling from the joint distribution
\[
	p(x_{1:T} \mid y_{1:T}),
\]
and approximating the associated marginal distributions
\[
	\{p(x_n \mid y_{1:T})\}_{n=1}^{T}.
\]
Although particle filtering techniques can be employed to tackle this problem, their performance often deteriorates as \(T\) becomes large. To mitigate these issues, several particle smoothing methods have been developed. Essentially, these methods rely on a particle implementation of either the forward filtering--backward smoothing (FFBS) formula or a generalized version of the two-filter smoothing formula.